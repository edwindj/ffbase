\documentclass[11pt, a4paper]{article}
\usepackage{amsfonts, amsmath, hanging, hyperref, natbib, parskip, times}
\usepackage[pdftex]{graphicx}
\hypersetup{
  colorlinks,
  linkcolor=blue,
  urlcolor=blue
}

\let\section=\subsubsection
\newcommand{\pkg}[1]{{\normalfont\fontseries{b}\selectfont #1}} 
\let\proglang=\textit
\let\code=\texttt 
\renewcommand{\title}[1]{\begin{center}{\bf \LARGE #1}\end{center}}
\newcommand{\affiliations}{\footnotesize}
\newcommand{\keywords}{\paragraph{Keywords:}}

\setlength{\topmargin}{-15mm}
\setlength{\oddsidemargin}{-2mm}
\setlength{\textwidth}{165mm}
\setlength{\textheight}{250mm}

\begin{document}
\pagestyle{empty}

\title{\pkg{ffbase}: statistical functions for large datasets}

\begin{center}
  {\bf Edwin de Jonge$^{1}$, Jan Wijffels$^{2}$}
\end{center}

\begin{affiliations}
1. Statistics Netherlands,  \href{mailto:e.dejonge@cbs.nl}{e.dejonge@cbs.nl} \\[-2pt]
2. Affiliation of Jan Wijffels \\[-2pt]
\end{affiliations}

\keywords Large datasets

\vskip 0.8cm

Statistical datasets used to be small, but nowadays \proglang{R}
\proglang{R} is an excellent environment for statistical analysis.



% Some suggestions: if you mention a programming language like
% \proglang{R}, typeset the language name with the {\tt \textbackslash
%   proglang\{\}} command.  If you mention an \proglang{R} function \code{foo},
% typeset the function name with the with the {\tt\textbackslash code\{\}}
% command.  If you mention an \proglang{R} package \pkg{fooPkg}, typeset
% the package name with the {\tt\textbackslash pkg\{\}} command.
% Abstracts should not exceed one page.  The page should not be numbered. 
% 

%% references: 
\nocite{ref1,ref2,ref3}
\bibliographystyle{chicago}
\bibliography{biblioExample}

%% references can alternatively be entered by hand
%\subsubsection*{References}

%\begin{hangparas}{.25in}{1}
%AuthorA (2007). Title of a web resource, \url{http://url/of/resource/}.

%AuthorC (2008a). Article example in proceedings. In \textit{useR! 2008, The R
%User Conference, (Dortmund, Germany)}, pp. 31--37.

%AuthorC (2008b). Title of an article. \textit{Journal name 6}, 13--17.
%\end{hangparas}

\end{document}
